\documentclass[11pt, fuzz]{article}

%----------------  Retrieving all the packages-----------------%
\usepackage[english]{babel} %change language 
\usepackage[T1]{fontenc}
\usepackage[utf8x]{inputenc}
\usepackage{amsmath}
\usepackage{fancyhdr}
\usepackage{graphicx}
\usepackage{tabularx}
\usepackage{url}
\usepackage{booktabs}
\usepackage[colorlinks=true,urlcolor=blue,linkcolor=black,citecolor=black]{hyperref}
\usepackage{fuzz}

\newcommand{\mono}[1]{{\ttfamily #1}}
\newcommand{\equwref}[1]{(\ref{#1})}


%---------------------- Hader -------------------------------%
\fancyhead[L]{Guillermo Cornejo Suárez \\ 2017239404}
\fancyhead[R]{Ingeniería de Software\\ Asignacion 1}
\fancyfoot[C]{\thepage}	 
\pagestyle{fancy}
\setlength{\headheight}{26pt}

               

%---------------------- Title --------------------------------%
\title{		
		\usefont{OT1}{bch}{b}{n}
        \Large \textsc{Tecnológico de Costa Rica} \\
		\normalsize \textsc{Maestría en computación} \\ 
		\normalsize \textsc{Ingeniería de Software} \\ [25pt]		
		\Huge Assignment 1 \\	
}
\author{
		\normalfont \Large   Guillermo Cornejo Suárez \\ 		
        \large		 		 2017239404 \\ 		        
}


\begin{document}

\maketitle

This assignment consist on developing a formal description for the Library Management System (LMS) explained in the reference book. A Library in composed of \emph{Book}s and a group of \emph{Person}s. 


\begin{zed}
    [Book, Person]
\end{zed}

The Library's books can be divided in books on shelves and books on loan. The system also requires a record of which person has a lent book. 


\begin{schema}{Library}
    on\_loan, on\_shelves, books: \power Book \\
    borrowers: \power Person                 \\
    lent\_to: Book \pfun Person              \\
\where
    on\_loan = \dom lent\_to                 \\
    \ran lent\_to \subset borrowers          \\
    books = on\_loan \cup on\_shelves        \\
    on\_loan \cap on\_shelves = \emptyset    
\end{schema}

The LMS is initialized as follows:

\begin{schema}{Init\_library}
    Library'
\where
    books' = \emptyset      \\
    borrowers' = \emptyset
\end{schema}

The type \emph{LMSResponse} defines the system responses. 

\begin{zed}
    LMSResponse ::= add\_book\_ok               \\
                  | add\_book\_already\_in      \\
                  | add\_borrower\_ok           \\
                  | add\_borrower\_already\_in  \\
                  | delete\_book\_ok            \\
                  | delete\_book\_not\_in       \\
                  | delete\_book\_on\_loan      \\
                  | delete\_borrower\_ok        \\
                  | delete\_borrower\_not\_in   \\
                  | delete\_borrower\_on\_loan  \\
                  | lend\_ok                    \\
                  | lend\_borrower\_not\_in     \\
                  | lend\_book\_not\_in         \\
                  | lend\_book\_on\_loan        \\
\end{zed}


The following schemas define the operations over the LMS. 

\section{Add a new book}

\begin{schema}{Add\_book\_ok}
    \Delta Library  \\
    b?: Book        \\
    r!: LMSResponse 
\where
    b? \notin books                      \\
    on\_shelves' = on\_shelves \cup \{b?\} \\
    r! = add\_book\_ok
\end{schema}

\begin{schema}{Add\_book\_already\_in}
    \Xi Library     \\
    b?: Book        \\
    r!: LMSResponse
\where
    b? \in books                \\
    r! = add\_book\_already\_in
\end{schema}

\begin{zed}
    Add\_book \defs Add\_book\_ok \lor Add\_book\_already\_in
\end{zed}


\section{Add a Borrower}

\begin{schema}{Add\_borrower\_ok}
    \Delta Library  \\
    b?: Person      \\
    r!: LMSResponse
\where
    b? \notin borrowers                \\
    borrowers' = borrowers \cup \{b?\} \\
    r! = add\_borrower\_ok    
\end{schema}

\begin{schema}{Add\_borrower\_already\_in}
    \Xi Library     \\
    b?: Person      \\
    r!: LMSResponse \\
\where
    b? \in borrowers                \\
    r! = add\_borrower\_already\_in
\end{schema}

\begin{zed}
    Add\_borrower \defs Add\_borrower\_ok \lor Add\_borrower\_already\_in
\end{zed}

\section{Delete a book}

\begin{schema}{Delete\_book\_ok}
    \Delta Library  \\
    b?: Book        \\
    r!: LMSResponse
\where
    b? \in books                                \\
    b? \in on\_shelves                          \\
    on\_shelves' = on\_shelves \setminus \{b?\} \\
    r! = delete\_book\_ok
\end{schema}

\begin{schema}{Delete\_book\_not\_in}
    \Xi Library     \\
    b?: Book        \\
    r!: LMSResponse
\where
    b? \notin books            \\
    r! = delete\_book\_not\_in \\
\end{schema}

\begin{schema}{Delete\_book\_on\_loan}
    \Xi Library     \\
    b?: Book        \\
    r!: LMSResponse
\where
    b? \in on\_loan             \\
    r! = delete\_book\_on\_loan \\
\end{schema}

\begin{zed}
    Delete\_book \defs Delete\_book\_ok \lor Delete\_book\_not\_in \lor Delete\_book\_on\_loan
\end{zed}


\section{Delete a borrower}


\begin{schema}{Delete\_borrower\_ok}
    \Delta Library  \\
    b?: Person      \\
    r!: LMSResponse \\
\where
    b? \in borrowers                        \\
    b? \notin \ran lent\_to                 \\
    borrowers' = borrowers \setminus \{b?\} \\
    r! = delete\_borrower\_ok               \\
\end{schema}

\begin{schema}{Delete\_borrower\_not\_in}
    \Xi Library     \\
    b?: Person      \\
    r!: LMSResponse \\
\where
    b? \notin borrowers            \\
    r! = delete\_borrower\_not\_in \\
\end{schema}

\begin{schema}{Delete\_borrower\_on\_loan}
    \Xi Library     \\
    b?: Person      \\
    r!: LMSResponse \\
\where
    b? \in borrowers                        \\
    b? \in \ran lent\_to                    \\
    r! = delete\_borrower\_on\_loan         \\
\end{schema}

\begin{zed}
    Delete\_borrower \defs Delete\_borrower\_ok \lor Delete\_borrower\_not\_in \lor \\ Delete\_borrower\_on\_loan
\end{zed}


\section{Lend a book}

\begin{schema}{Lend\_ok}

\end{schema}





\end{document}




