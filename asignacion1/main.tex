\documentclass[11pt, fuzz]{article}

%----------------  Retrieving all the packages-----------------%
\usepackage[english]{babel} %change language 
\usepackage[T1]{fontenc}
\usepackage[utf8x]{inputenc}
\usepackage{amsmath}
\usepackage{fancyhdr}
\usepackage{graphicx}
\usepackage{tabularx}
\usepackage{url}
\usepackage{booktabs}
\usepackage[colorlinks=true,urlcolor=blue,linkcolor=black,citecolor=black]{hyperref}
\usepackage{fuzz}

\newcommand{\mono}[1]{{\ttfamily #1}}
\newcommand{\equwref}[1]{(\ref{#1})}


%---------------------- Hader -------------------------------%
\fancyhead[L]{Guillermo Cornejo Suárez \\ 2017239404}
\fancyhead[R]{Ingeniería de Software\\ Asignacion 1}
\fancyfoot[C]{\thepage}	 
\pagestyle{fancy}
\setlength{\headheight}{26pt}

               

%---------------------- Title --------------------------------%
\title{		
		\usefont{OT1}{bch}{b}{n}
        \Large \textsc{Tecnológico de Costa Rica} \\
		\normalsize \textsc{Maestría en computación} \\ 
		\normalsize \textsc{Ingeniería de Software} \\ [25pt]		
		\Huge Assignment 1 \\	
}
\author{
		\normalfont \Large   Guillermo Cornejo Suárez \\ 		
        \large		 		 2017239404 \\ 		        
}


\begin{document}

\maketitle

This assignment consist on developing a formal description for the Library Management System (LMS) explained in the reference book. A Library in composed of \emph{Book}s and a group of \emph{Person}s. 


\begin{zed}
    [Book, Person]
\end{zed}

The Library's books can be divided in books on shelves and books on loan. The system also requires a record of which person has a lent book. 


\begin{schema}{Library}
    on\_loan, on\_shelves, books: \power Book \\
    borrowers: \power Person                 \\
    lent\_to: Book \pfun Person              \\
\where
    on\_loan = \dom lent\_to                 \\
    \ran lent\_to \subset borrowers          \\
    books = on\_loan \cup on\_shelves        \\
    on\_loan \cap on\_shelves = \emptyset    
\end{schema}

The LMS is initialized as follows:

\begin{schema}{Init\_library}
    Library'
\where
    books' = \emptyset      \\
    borrowers' = \emptyset
\end{schema}

The type \emph{LMSResponse} defines the system responses. 

\begin{zed}
    LMSResponse ::= add\_book\_ok               \\
                  | add\_book\_already\_in      \\
                  | add\_borrower\_ok           \\
                  | add\_borrower\_already\_in  \\
                  | delete\_book\_ok            \\
                  | delete\_book\_not\_in       \\
                  | delete\_book\_on\_loan      \\
                  | delete\_borrower\_ok        \\
                  | delete\_borrower\_not\_in   \\
                  | delete\_borrower\_on\_loan  \\
                  | lend\_ok                    \\
                  | lend\_borrower\_not\_in     \\
                  | lend\_book\_not\_in         \\
                  | lend\_book\_on\_loan        \\
                  | return\_book\_ok            \\
                  | return\_book\_not\_in       \\
                  | return\_book\_not\_loan     \\
                  | enquire\_in                 \\
                  | enquire\_not\_in            \\
                  | enquire\_on\_loan           \\
                  | delete\_book\_reserved      \\
                  | lend\_book\_reserved        \\
                  | reserve\_title\_not\_in     \\
                  | reserve\_title\_not\_borrower     \\
                  | reserve\_title\_ok          \\
\end{zed}


The following schemas define the operations over the LMS. 

\section{Add a new book}

To add a new book to the library, it must not be in the library. 

\begin{schema}{Add\_book\_ok}
    \Delta Library  \\
    b?: Book        \\
    r!: LMSResponse 
\where
    b? \notin books                      \\
    on\_shelves' = on\_shelves \cup \{b?\} \\
    r! = add\_book\_ok
\end{schema}

\begin{schema}{Add\_book\_already\_in}
    \Xi Library     \\
    b?: Book        \\
    r!: LMSResponse
\where
    b? \in books                \\
    r! = add\_book\_already\_in
\end{schema}

\begin{zed}
    Add\_book \defs Add\_book\_ok \lor Add\_book\_already\_in
\end{zed}


\section{Add a Borrower}

Similar to adding a new book, the new borrower must not be in the library system. 

\begin{schema}{Add\_borrower\_ok}
    \Delta Library  \\
    b?: Person      \\
    r!: LMSResponse
\where
    b? \notin borrowers                \\
    borrowers' = borrowers \cup \{b?\} \\
    r! = add\_borrower\_ok    
\end{schema}

\begin{schema}{Add\_borrower\_already\_in}
    \Xi Library     \\
    b?: Person      \\
    r!: LMSResponse \\
\where
    b? \in borrowers                \\
    r! = add\_borrower\_already\_in
\end{schema}

\begin{zed}
    Add\_borrower \defs Add\_borrower\_ok \lor Add\_borrower\_already\_in
\end{zed}

\section{Delete a book}

To delete a book, the library must own the book and it be on the shelfs.

\begin{schema}{Delete\_book\_ok}
    \Delta Library  \\
    b?: Book        \\
    r!: LMSResponse
\where
    b? \in books                                \\
    b? \in on\_shelves                          \\
    on\_shelves' = on\_shelves \setminus \{b?\} \\
    r! = delete\_book\_ok
\end{schema}

\begin{schema}{Delete\_book\_not\_in}
    \Xi Library     \\
    b?: Book        \\
    r!: LMSResponse
\where
    b? \notin books            \\
    r! = delete\_book\_not\_in \\
\end{schema}

\begin{schema}{Delete\_book\_on\_loan}
    \Xi Library     \\
    b?: Book        \\
    r!: LMSResponse
\where
    b? \in on\_loan             \\
    r! = delete\_book\_on\_loan \\
\end{schema}

\begin{zed}
    Delete\_book \defs Delete\_book\_ok \lor Delete\_book\_not\_in \lor Delete\_book\_on\_loan
\end{zed}


\section{Delete a borrower}

To delete a borrower, it must be in the borrowers list and she must not have a lend book from the library. 

\begin{schema}{Delete\_borrower\_ok}
    \Delta Library  \\
    b?: Person      \\
    r!: LMSResponse \\
\where
    b? \in borrowers                        \\
    b? \notin \ran lent\_to                 \\
    borrowers' = borrowers \setminus \{b?\} \\
    r! = delete\_borrower\_ok               \\
\end{schema}

\begin{schema}{Delete\_borrower\_not\_in}
    \Xi Library     \\
    b?: Person      \\
    r!: LMSResponse \\
\where
    b? \notin borrowers            \\
    r! = delete\_borrower\_not\_in \\
\end{schema}

\begin{schema}{Delete\_borrower\_on\_loan}
    \Xi Library     \\
    b?: Person      \\
    r!: LMSResponse \\
\where
    b? \in borrowers                        \\
    b? \in \ran lent\_to                    \\
    r! = delete\_borrower\_on\_loan         \\
\end{schema}

\begin{zed}
    Delete\_borrower \defs Delete\_borrower\_ok \lor Delete\_borrower\_not\_in \lor \\ Delete\_borrower\_on\_loan
\end{zed}


\section{Lend a book}

To lend a book, it must be on the shelfs and the person must be on the list of borrowers. 

\begin{schema}{Lend\_ok}
    \Delta Library  \\
    b?: Book        \\
    p?: Person      \\
    r!: LMSResponse \\
\where
    b? \in on\_shelves                          \\
    p? \in borrowers                            \\
    lent\_to' = \{b? \mapsto p?\} \cup lent\_to \\
    r! = lend\_ok                               \\
\end{schema}

\begin{schema}{Lend\_borrower\_not\_in}
    \Xi Library  \\
    b?: Book        \\
    p?: Person      \\
    r!: LMSResponse \\
\where
    p? \notin borrowers          \\
    r! = lend\_borrower\_not\_in \\
\end{schema}

\begin{schema}{Lend\_book\_not\_in}
    \Xi Library  \\
    b?: Book        \\
    p?: Person      \\
    r!: LMSResponse \\
\where
    b? \notin books          \\
    r! = lend\_book\_not\_in \\
\end{schema}

\begin{schema}{Lend\_book\_on\_loan}
    \Xi Library  \\
    b?: Book        \\
    p?: Person      \\
    r!: LMSResponse \\
\where
    b? \in on\_loan          \\
    r! = lend\_book\_on\_loan \\
\end{schema}


\begin{zed}
    Lend\_book \defs Lend\_ok \lor Lend\_borrower\_not\_in \lor Lend\_book\_not\_in \lor \\ Lend\_book\_on\_loan
\end{zed}

\section{Return a book}

To return a book, it must be on the list of loan books

\begin{schema}{Return\_book\_ok}
    \Delta Library  \\
    b?: Book        \\
    r!: LMSResponse \\
\where
    b? \in on\_loan                        \\
    on\_loan' = on\_loan \setminus \{b?\}  \\
    lent\_to' = \{b?\} \ndres lent\_to     \\
    on\_shelves' = on\_shelves \cup \{b?\} \\
    r! = return\_book\_ok                   \\
\end{schema}

\begin{schema}{Return\_book\_not\_in}
    \Xi Library     \\
    b?: Book        \\
    r!: LMSResponse \\
\where
    b? \notin books                        \\
    r! = return\_book\_not\_in              \\
\end{schema}

\begin{schema}{Return\_book\_not\_loan}
    \Xi Library     \\
    b?: Book        \\
    r!: LMSResponse \\
\where
    b? \notin on\_loan          \\
    r! = return\_book\_not\_loan \\
\end{schema}

\begin{zed}
    Return\_book \defs Return\_book\_ok \lor Return\_book\_not\_in \lor \\ Return\_book\_not\_loan
\end{zed}


\section{Adding the Titles extension}

This extension creates an abstractions for titles, i.e. the information relative to a book, like author and press, but preserving independence from the physical book. The physical book is already represented by the Book abstraction. 

\begin{zed}
    [Title]
\end{zed}

\begin{schema}{Titles\_of\_books}
    books: \power Book          \\
    titles: \power Title        \\
    title\_of: Book \pfun Title 
\where
    books = \dom title\_of \\
    titles = \ran title\_of \\
\end{schema}

\begin{zed}
    Library\_phase2 \defs Library \land Titles\_of\_books
\end{zed}


\section{Add book with title}

To add a new title, similar to adding a new book, it must not be registered in the system. Also, every title must be associated with a physical book within the library. 

\begin{schema}{Add\_book\_to\_titles\_ok}
    \Xi Library              \\
    \Delta Titles\_of\_books \\
    b?: Book                 \\
    t?: Title                \\
\where
    b? \notin books                             \\
    title\_of' = title\_of \cup \{b? \mapsto t?\} \\
\end{schema}

\begin{zed}
    Add\_book\_ok\_p2 \defs Add\_book\_ok \land Add\_book\_to\_titles\_ok
\also
    Add\_book\_already\_in\_p2 \defs Add\_book\_already\_in \land \Xi Titles\_of\_books
\also
    Add\_book\_p2 \defs Add\_book\_ok\_p2 \lor Add\_book\_already\_in\_p2 
\end{zed}


\section{Enquire about a book}

A borrower can retrieve the title's information by giving the system a book. Hence, the user can know if the title is owned by the library and, if so, if it's on the shelves or on loan. 

\begin{schema}{Enquire\_in}
    \Xi Library\_phase2  \\
    b?: Book              \\
    r!: LMSResponse       \\
\where
    b? \in on\_shelves    \\
    r! = enquire\_in      \\
\end{schema}

\begin{schema}{Enquire\_not\_in}
    \Xi Library\_phase2  \\
    b?: Book              \\
    r!: LMSResponse       \\
\where
    b? \notin books       \\
    r! = enquire\_not\_in \\
\end{schema}

\begin{schema}{Enquire\_on\_loan}
    \Xi Library\_phase2  \\
    b?: Book              \\
    r!: LMSResponse       \\
\where
    b? \in on\_loan       \\
    r! = enquire\_on\_loan\\
\end{schema}

\begin{zed}
    Enquire\_book \defs Enquire\_in \land Enquire\_not\_in \land Enquire\_on\_loan
\end{zed}


\section{Reservations}

A borrower can reserve a book owned by the library. By doing so, no other borrower will be able to take the book on loan before her. 

\begin{schema}{Reservation}
    Library\_phase2                        \\
    reservation: Title \pfun \iseq Person
\where
    \dom reservation \subseteq titles  \\
    \bigcup \{t: \dom reservation @ \ran(reservation(t))\} \subseteq borrowers
\end{schema}

\begin{zed}
Library\_phase3 \defs Library\_phase2 \land Reservation
\end{zed}

\section{Delete a book with reservation}

To delete a book, considering the new reservation abstraction, the system must first check if a borrower had reserved it. 

\begin{schema}{Delete\_book\_reserv\_reserved}
    \Xi Library\_phase3 \\
    b?: Book            \\
    r!: LMSResponse
\where
    title\_of(b?) \in \dom reservation \\
    r! = delete\_book\_reserved
\end{schema}

\begin{zed}
    Delete\_book\_p3 \defs Delete\_book \lor Delete\_book\_reserv\_reserved
\end{zed}

\section{Delete Borrower with reservation}

To delete a borrower, considering the new reservation abstraction, the system must remove the borrower from all the reservation lists. 

\begin{schema}{Delete\_borrower\_reservation}
    \Delta Library\_phase3 \\
    b?: Person             \\
    r!: LMSResponse 
\where
    reservation = (\lambda title:Title | title \in \dom reservation @ reservation(title) \filter \{b?\}) 
\end{schema}


\begin{zed}
Delete\_borrower\_p3 \defs (Delete\_borrower\_ok \land Delete\_borrower\_reservation) \\ \lor Delete\_borrower\_not\_in \lor Delete\_borrower\_on\_loan
\end{zed}


\section{Lend a book with reservation}

To lend a book, considering the new reservation abstraction, the system must check if the borrower requesting the book is the fist on the reservation queue. 

\begin{schema}{Lend\_book\_reserved}
    \Xi Library\_phase3 \\
    b?: Book            \\
    p?: Person          \\
    r!: LMSResponse     \\
\where
    reservation(title\_of(b?)) \neq \emptyset \\
    p? \neq head~(reservation(title\_of(b?))) \\
    r! = lend\_book\_reserved
\end{schema}    

\begin{schema}{Lend\_book\_reserved\_ok}
    \Delta Library\_phase3 \\
    b?: Book               \\
    p?: Person             \\
    r!: LMSResponse        \\
\where    
    reservation(title\_of(b?)) = \emptyset \lor p? = head~(reservation(title\_of(b?))) \\
    reservation' = reservation \oplus \{(title\_of(b?)) \mapsto (tail~(reservation(title\_of(b?))))\} \\
    Lend\_ok               
\end{schema}

\begin{zed}
Lend\_book\_reservation \defs Lend\_borrower\_not\_in \lor \\ Lend\_book\_not\_in \lor Lend\_book\_on\_loan \lor Lend\_book\_reserved \lor \\ Lend\_book\_reserved\_ok
\end{zed}

\section{Reserve a title}

To reserve a title, the title must be owned by the library and the borrower must not have reserved the title previously. 

\begin{schema}{Reserve\_title\_not\_in}
    \Xi Library\_phase3 \\
    t?: Title           \\
    p?: Person          \\
    r!: LMSResponse     \\
\where
    t? \notin titles \\
    r! = reserve\_title\_not\_in \\
\end{schema}

\begin{schema}{Reserve\_title\_not\_borrower}
    \Xi Library\_phase3 \\
    t?: Title           \\
    p?: Person          \\
    r!: LMSResponse     \\
\where
    p? \notin borrowers                \\
    r! = reserve\_title\_not\_borrower
\end{schema}

\begin{schema}{Reserve\_title\_already\_reserved}
    \Xi Library\_phase3 \\
    t?: Title           \\
    p?: Person          \\
    r!: LMSResponse     \\
\where
    p? \in \ran(reservation(t?))    \\
    r! = reserve\_title\_not\_borrower
\end{schema}

\begin{schema}{Reserve\_title\_ok}
    \Delta Library\_phase3 \\
    t?: Title              \\
    p?: Person             \\
    r!: LMSResponse        \\
\where    
    p? \in borrowers                \\
    t? \in titles                   \\  
    p? \notin \ran(reservation(t?))  \\

    reservation' = reservation \oplus \{t? \mapsto (reservation(t?) \cat \langle p? \rangle)\} \\
    r! = reserve\_title\_ok
\end{schema}


\begin{zed}
Reserve\_title \defs Reserve\_title\_not\_in \lor Reserve\_title\_not\_borrower \\ \lor Reserve\_title\_already\_reserved \lor Reserve\_title\_ok 
\end{zed}

\end{document}

